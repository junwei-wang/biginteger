%%%%%%%%%%%%%%%%%%%%%%%%%%%%%%%%%%%%%%%%%%%%%%%%%%%%%%
% Filename:      implementation.tex
% Author:        Junwei Wang(wakemecn@gmail.com)
% Last Modified: 2013-12-30 21:50
% Description:
%%%%%%%%%%%%%%%%%%%%%%%%%%%%%%%%%%%%%%%%%%%%%%%%%%%%%%
\section{Our Implementation}
In this section, we describe our implementation of group structure on elliptic
curves, simplified SWU algorithm and Simple Password Exponential Key Exchange
in detail.
\subsection{Group Operations on Elliptic Curve}
Let $E: y^2 = x^3  - 3x + b\ \texttt{mod}\ p$ is an elliptic curve, where $q$ is
the group order based on $E$, point $G = (G_x, G_y)$ is the generator of the
group. Then we define the identity of the group, i.e., the ``point at
infinity" $\mathcal{O}$ equal to $qP$. Because we cannot represent
$\mathcal{O}$ in tradition two-dimension coordinates, this definition is of 
great convenience to computation on the group. According to the definition of the group on
elliptic curves,
the inverse of any point $P$ is the point $P'$, that has same x-coordinate 
with $P'$, except that the inverse of ``point at infinity" $\mathcal{O}$ is 
itself. 
\par The implementation of addition operation in the group follows the
Equation~\ref{groupPlus} directly. The multiplication operation is implemented in the 
\emph{double-and-add} methods, which is as efficient as
\emph{square-and-multiple} on multiplicative groups.
\begin{algorithm} 
\caption{Double-and-Add Algorithm}
  \begin{algorithmic}[1]
    \Require Point $P$ on elliptic curve $E$, positive integer $m =
(m_km_{k-1}...m_0)_2$ where $m_k = 1$
    \Ensure Point $Q$ on $E$

    \For {$i$ from $k-1$ downto $0$} 
      \State $Q \gets 2P$
      \If {$m_i = 1$}
         \State $Q \gets Q + P$
      \EndIf
    \EndFor
    \State \Return $Q$
  \end{algorithmic}
\end{algorithm}
For more detail, please refer to Appendix~\ref{appA}.


\subsection{Simplified Shallue-Woestijne-Ulas Algorithm}
The implementation of simplified SWU Algorithm was first published in 
\cite{brier2010efficient}, which is exactly the same as described in
Algorithm~\ref{sswu}. We discover that there exists some repeated computation is the
second part (i.e., selection part) of the implementation, that is
$\texttt{is\_square}(h_2)$ (i.e., $h_2 ^ {(p-1)/2}$ as Theorem~\ref{square})
and $h_2 ^ {(p+1)/4}$ are both powers of $h_2$.
For any $p = 3\ \texttt{mod}\ 4$, we observe that $(p-3)/4$ is an integer
and
\begin{equation}
\label{h2}
\begin{array}{lll}
h_2^{(p-1)/2} & = & (h_2^{(p-3)/4})^2 * h_2, \\
h_2^{(p+1)/4} & = & h_2^{(p-3)/4} * h_2.
\end{array}
\end{equation}
Hence, in the implementation, we compute $h_2^{(p-3)/4}$ firstly, and then
compute $h_2^{(p-1)/2}$ and $h_2^{(p+1)/4}$ according to Equation~\ref{h2}. 
By this approach, we save nearly half of the computation time when $h_2$ is a
square, in other words, when $(x_2, h_2^{(p+1)/4})$ is a point on specified elliptic curve.

\par When $h_2$ is not a square, that is, $h_3$ is a square, according to
Equation~\ref{u}, we observe that
$(t^3g(X_2(t)))^2 = -g(X_2(t)) \cdot g(X_3(t))$,
in other words,
$(t^3 \cdot h_2)^2 = -h_2 \cdot h_3$. 
Therefore,
\begin{equation}
h_3  =  -t^6h_2,
\end{equation}
and
\begin{equation}
\label{h3}
\begin{array}{lll}
h_3^{(p-3)/4} & = & (-1)^{(p-3)/4} \cdot (t^6)^{(p-3)/4} \cdot h_2^{(p-3)/4}\\
& = & (-1)^{(p-3)/4} \cdot (t^3)^{(p-1)/2} \cdot t^{-3} \cdot h_2^{(p-3)/4},\\
h_3^{(p+1)/4} & = & (-1)^{(p-3)/4} \cdot (t^3)^{(p-1)/2} \cdot t^{-3} \cdot
h_2^{(p-3)/4} \cdot h_3.
\end{array}
\end{equation}
Since $|(-1)^{(p-3)/4} \cdot (t^3)^{(p-1)/2}|=1$, both 
$t^{-3} \cdot h_2^{(p-3/4)} \cdot h_3$ and
$-t^{-3} \cdot h_2^{(p-3/4)} \cdot h_3$ are the roots of $h_3$. 
By this approach, we save nearly half of the computation time when $h_3$ is  a
square, in other words, when $(x_3, h_3^{(p+1)/4})$ is a point on specified elliptic curve.

\par Our efficient algorithm is showed below.
\begin{algorithm} 
\caption{Efficient simplified SWU algorithm \cite{brier2010efficient}}
\label{sswu}
  \begin{algorithmic}[1]
    \Require $\mathbb{F}_q$ such that $q \equiv 3\ (\texttt{mod}\ 4)$, parameters
$a$, $b$ and input $t \in \mathbb{F}_q$
    \Ensure $(x, y) \in E_{a,b}(\mathbb{F}_q)$ where $E_{a,b}: y^2 = x^3 + ax +
b$

    \State $\alpha \gets -t^2$
    \State $X_2 \gets \frac{-b}{a}(1+\frac{1}{\alpha^2+\alpha})$
    \State $X_3 \gets \alpha \cdot X_2$
    \State $h_2 \gets (X_2)^3 + a\cdot X_2 + b$
    \State $h_3 \gets (X_3)^3 + a\cdot X_3 + b$
    \State $t_2 \gets h_2^{(p-3)/4}$
    \State $t_3 \gets t^{-3}$
    \If{$h_2$ is a square}
      \State \Return $(X_2, t_2 \cdot h_2)$
    \Else
      \State \Return $(X_3, t_2 \cdot t_3 \cdot h_3)$
    \EndIf
  \end{algorithmic}
\end{algorithm}
For more detail, please refer to Appendix~\ref{appB}.

\subsection{Simple Password Exponential Key Exchange}
In the implementation of SPEKE, we use $\mathrm{SHA256(\cdot)}\
\texttt{mod}\ q$ \cite{fips2002180} as the hash function into $\mathbb{F}_q$,
where $q$ is 192 length. The reminder of the implementation is 
trivial.  For more detail, please refer to Appendix~\ref{appC}.
